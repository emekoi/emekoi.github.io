% change research experience to work experience
% add ta-ing to work experience
% remove some of the less impressive stuff? zig? brainfuck? unlambda?
% add cs81 compiler?

\documentclass[a4paper]{article}
\usepackage{resume}

\begin{document}
    %==== Contact Info ====%
    \contact{Emeka Nkurumeh}{Pasadena, CA}{\href{https://emekoi.github.io/}{https://emekoi.github.io/}}{\href{mailto:e.nk@caltech.edu}{e.nk@caltech.edu}}

    %==== Education ====%
    \header{Education}
    \begin{school}{California Institute of Technology}{Pasadena, CA}{3.4}{Sep 2021--June 2025}
        % \item Programming Fundamentals, Methods, and Software Design (CS 1--4), Decidability and Tractability (CS 21), Algorithms (CS 38), Operating Systems (CS 124), Interactive Theorem Proving (CS 128), Programming Languages (CS 131), Compilers (CS 164), Computer Graphics Laboratory (CS 171)

        % \item Decidability and Tractability (CS 21), Algorithms (CS 38), Interactive Theorem Proving (CS 128), Programming Languages (CS 131), Compilers (CS 164)
        % \item Calculus and Linear Algebra (Ma 1), Differential Equations (Ma 2), Abstract Algebra (Ma 5), Discrete Math (Ma 6), Introduction to Geometry and Topology (Ma 109), Computability Theory (Ma 117), Graduate Probability (Ma 140)

        \item Decidability and Tractability (CS 21), Algorithms (CS 38), Interactive Theorem Proving (CS 128), Programming Languages (CS 131), Compilers (CS 164)
        \item Calculus and Linear Algebra (Ma 1), Differential Equations (Ma 2), Abstract Algebra (Ma 5), Discrete Math (Ma 6), Computability Theory (Ma 117)

    \end{school}

    \header{Awards}
    \award{Gordon McClure Memorial Communications Prize in History}{June 2024}{Caltech}
    \award{George W. Housner Student Discovery Fund}{April 2024}{Caltech}
    \award{OPLSS Grant}{April 2024}{University of Oregon}
    \award{PLMW @ POPL 2024}{Jan 2024}{ACM SIGPLAN/NSF}
    \award{Gates Scholar}{Apr 2021}{The Bill \& Melinda Gates Foundation}
    \award{QuestBridge National College Match Scholar}{Dec 2020}{QuestBridge}

    \header{Skills}
    \begin{skill}{Programming Languages}
      \begin{itemize}[nosep]
        \item Haskell, OCaml, Coq, C, Agda, Java, Python, x86/x86\_64 Assembly, Lua, Zig, \LaTeX{}
      \end{itemize}
    \end{skill}
    \vspace{-8pt}
    \begin{skill}{Unix Operating Systems}
      \begin{itemize}[nosep]
        \item Bash, Fish, Vim, Emacs, Git, gdb, Linux, various Unix command line programs
     \end{itemize}
    \end{skill}
    \vspace{-5pt}

    \header{Research Experience}
    \begin{activity*}{Formally Verified CBPV Compiler}{Caltech}{Jan 2024--present}
      \begin{itemize}[topsep=5pt, partopsep=0pt, itemsep=-1pt]
        \item Formalized the type system and semantics of a call-by-push-value calculus in Agda and implemented a sound and complete type checker.
        \item Implemented a reference interpreter and started work on compiling to a small stack machine with the goal of proving them equivalent.
      \end{itemize}
    \end{activity*}

    \begin{activity*}{Stream Type Transformers}{University of Pennsylvania}{June 2023--Aug 2023}
      \begin{itemize}[topsep=5pt, partopsep=0pt, itemsep=-1pt]
        \item Worked with Benjamin C. Pierce and Joseph W. Cutler on efficient type inference for a calculus of typed stream transformers based on ordered bunched logics, known as `Stream Types'.
        \item Reformulated the existing system as a labeled calculus and developed a novel polynomial-time type checking algorithm in Haskell, improving over a previous exponential time algorithm.
      \end{itemize}
    \end{activity*}

    \begin{activity*}{Site Percolation on 2D Lattices}{OSSM}{Aug 2020--Dec 2020}
      \begin{itemize}[topsep=5pt, partopsep=0pt, itemsep=-1pt]
        \item Worked with Jayanta Rudra on creating and optimizing a visualizer for site percolation on 2D lattices for use in calculating critical thresholds for spanning clusters.
        \item Started work on an interactive visualizer for site percolation in higher dimensional lattices.
      \end{itemize}
    \end{activity*}

    \header{Work Experience}
    \begin{activity*}{Teaching Assistant}{Caltech}{Jan 2024--present}
      \begin{itemize}[topsep=5pt, partopsep=0pt, itemsep=-1pt]
        \item Graded assignments and led TA sessions.
      \end{itemize}
    \end{activity*}

    \header{Projects and Extracurriculars}
    \begin{activity}{Zig Programming Language Project}{Sept 2018--Sept 2023}
      \begin{itemize}[topsep=5pt, partopsep=0pt, itemsep=-1pt]
        \item Contributed to standard library, various stages of the bootstrapping compiler, and improved support for non-MSVC based build environments on Windows.
        \item Reported several bugs and made multiple influential language proposals.
      \end{itemize}
    \end{activity}

    % \begin{activity}{Brainf*ck Compiler}{Dec 2021}
    %   \begin{itemize}[topsep=5pt, partopsep=0pt, itemsep=-1pt]
    %     \item Implemented a dependency-free Brainf*ck interpreter and compiler with support for outputting ELF executables with no dependency on any system libraries.
    %     \item Implemented several optimizations leading to a 2-5x performance increase over the baseline compiler and a reduction of the number of syscalls needed for IO-bound programs.
    %   \end{itemize}
    % \end{activity}

    \begin{activity}{Unlambda Interpreter}{Mar 2022}
      \begin{itemize}[topsep=5pt, partopsep=0pt, itemsep=-1pt]
        \item Wrote an Unlambda interpreter using a modified CEK machine to capture and restore continuations.
        \item Started work on a graph rewriting based approach for interpretation of combinator-based code and compilation to imperative languages such as x86\_64 assembly.
      \end{itemize}
    \end{activity}

    % \begin{activity}{Bidirectional Type Checkers}{Jul 2022--Mar 2023}
    %   Implemented type checkers for the bidirectional type systems presented in
    %   \begin{itemize}[topsep=5pt, partopsep=0pt, itemsep=-1pt]
    %     \item \textit{A Mechanical Formalization of Higher-Ranked Polymorphic Type Inference}
    %     \item \textit{Complete and Easy Bidirectional Typechecking for Higher-Rank Polymorphism}
    %     \item \textit{Let Arguments Go First}
    %   \end{itemize}
    % \end{activity}
\end{document}
